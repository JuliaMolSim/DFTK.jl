\documentclass[a4paper]{amsart}
%% Language and font encodings
\usepackage[english]{babel}
\usepackage[utf8]{inputenc}
% \usepackage[T1]{fontenc}
\usepackage{amsmath,amssymb,amsfonts}
% \usepackage{lmodern}
% \usepackage{fullpage}

%% Sets page size and margins
\usepackage[a4paper,top=3cm,bottom=2cm,left=3cm,right=3cm,marginparwidth=1.75cm]{geometry}

%% Useful packages
\usepackage{amsmath,amsthm}
\usepackage{graphicx}
\usepackage[colorinlistoftodos]{todonotes}
\usepackage[colorlinks=true, allcolors=blue]{hyperref}
\def\C{{\mathbb C}}
\def\R{{\mathbb R}}
\renewcommand{\Re}{{\rm Re} \, }
\renewcommand{\Im}{{\rm Im} \, }
\newcommand{\I}{i}
\newcommand{\Op}[1]{#1}
\newcommand*{\bra}[1]{\left\langle#1\right|}
\newcommand*{\ket}[1]{\left|#1\right\rangle}
% 
% Bra and kets with something in the middle.
% They do have to come as a pair!
\newcommand*{\mbra}[1]{\left\langle#1\middle|}
  \newcommand*{\mket}[1]{\middle|#1\right\rangle}
% 
% Ketbra and braket together:
\newcommand*{\braket}[2]{\left\langle#1\middle|#2\right\rangle}
\newcommand*{\ketbra}[2]{\left| #1 \middle\rangle \middle\langle #2 \right|}
% 
% Commutator and anticommutator:
\newcommand*{\comm}[2]{\ensuremath{ \left[ {#1}, {#2} \right] }}
\newcommand*{\acomm}[2]{\ensuremath{ \left\{ {#1}, {#2} \right\} }}
% 
% Creation and anihilation operators:
\newcommand*{\cop}[1]{\ensuremath{\op{c}_{#1}^{\dagger}}}
\newcommand*{\aop}[1]{\ensuremath{\op{c}_{#1}}}
\newcommand{\norm}[1]{\|#1\|}
\renewcommand{\vec}[1]{{\mathbf{#1}}}
\newcommand*{\uvec}[1]{\ensuremath{\hat{{\boldsymbol{#1}}}}}	%unit vector
\newcommand{\D}{d}
\DeclareMathOperator{\erf}{erf}
\DeclareMathOperator{\erfc}{erfc}
\DeclareMathOperator{\sinc}{sinc}

\newtheorem*{theorem}{Theorem}
\newtheorem*{example}{Example}
\DeclareMathOperator{\argmin}{argmin}
% \DeclareMathOperator{\min}{min}

\begin{document}
\title{Definitions, conventions and useful facts}
\author{Michael F. Herbst, Antoine Levitt}
\maketitle

\section{Terminology}
\begin{description}
\item[$\vec{G}$] Vector on the reciprocal lattice $\mathcal{R}^\ast$
\item[$\vec{k}$] Vector inside the first Brillouin zone
\item[$\vec{q}$] Reciprocal-space vector $\vec{q} = \vec{k} + \vec{G}$
\item[$\vec{R}$] Vector on the lattice $\mathcal{R}$
\item[$\vec{r}$] Vector inside the unit cell
\item[$\vec{x}$] Real-space vector, $\vec{x} = \vec{r} + \vec{R}$
\item[$\Omega$] Unit cell / unit cell volume
\item[$e_{\vec{G}}$] Normalized plane wave
  \[ e_{\vec{G}} = \frac1{\sqrt{\Omega}} \exp(\I \vec{G} \cdot \vec{r})\]
\item[$\Op{T}_{\vec{R}}$] Lattice translation operator
  \[ \Op{T}_{\vec{R}} u(\vec{x}) = u(\vec{x} -
    \vec{R}) \]
\item[$Y_l^m$] complex spherical harmonics
\item[$Y_{lm}$] real spherical harmonics
\item[$\vec x = x \uvec x$] Separation of a vector into its radial and
  angular part
\item[$j_{l}$] Spherical Bessel functions. $j_{0}(x) = \frac{\sin x}{x}$
\end{description}

\section{Conventions and useful formulas}
\begin{itemize}
\item The Fourier transform is
  \begin{align*}
    \widehat{f}(\vec q) = \int_{\R^{3}} e^{-i\vec q \cdot \vec x} d\vec x
  \end{align*}
\item Plane wave expansion
  \begin{align*}
    e^{\I \vec{q} \cdot \vec{r}} =
  4 \pi \sum_{l = 0}^\infty \sum_{m = -l}^l
  \I^l j_l(q r) Y_l^m(\uvec{q}) Y_l^{m\ast}(\uvec{r})
\end{align*}
\item Spherical harmonics orthogonality
  \[
    \int_{\mathbb{S}^2} Y_l^{m*}(\uvec{r})Y_{l'}^{m'}(\uvec{r}) \D \uvec{r}
  = \delta_{l,l'} \delta_{m,m'}
\]
This also holds true for real spherical harmonics.
\item Fourier transforms of centered functions.

  If 
\[ f(\vec{x}) = R(x) Y_l^m(\uvec{x}),\]
then
\begin{align*}
  \hat f(\vec q)
  &= \int_{\R^3} R(x) Y_{l}^{m}(\uvec x) e^{-\I \vec{q} \cdot \vec{x}} \D\vec{x} \\
  &= \sum_{l = 0}^\infty 4 \pi \I^l 
  \sum_{m = -l}^l \int_{\R^3}
  R(x) j_{l'}(q x)Y_{l'}^{m'}(-\uvec{q}) Y_{l}^{m}(\uvec{x})
   Y_{l'}^{m'\ast}(\uvec{x})
  \D\vec{x} \\
  &= 4 \pi Y_{l}^{m}(-\uvec{q}) \I^{l}
  \int_{\R^+} x^2 R(x) \ j_{l}(q x)
  \
   \D x,
 \end{align*}
This also holds true for real spherical harmonics.
\end{itemize}
\section{The discretization}
The periodic part of Bloch waves is discretized in a set of normalized
plane waves $e_{\vec G}$:
\begin{align*}
  \psi_{k}(x) &= e^{\I \vec k \cdot \vec x} u_{\vec k}(\vec r)\\
  &= \sum_{\vec G \in \mathcal R^{*}} c_{G}  e^{\I \vec k \cdot \vec x} e_{\vec G}(\vec r)
\end{align*}
The $c_{G}$ are $\ell^{2}$-normalized. The summation is truncated to a
``spherical'' basis set
\begin{align*}
  S_{k} = \left\{G \in \mathcal R^{*}, \frac 1 2 |\vec k+\vec G|^{2} \le E_{\rm cut}\right\}
\end{align*}
Densities involve terms like $|\psi_{k}|^{2} = |u_{k}|^{2}$ and
therefore products $e_{-G} e_{G'}$ for $G, G'$ in $X_{k}$. To
represent these we use a ``cubic'' basis set large enough to contain
the set $\{G-G', G, G' \in S_{k}\}$. We can obtain the decomposition
of densities on the $e_{G}$ basis by a convolution, which can be
performed efficiently with FFTs. Potentials are discretized on this
same set. The normalization conventions used in the code is that
quantities stored in reciprocal space are coefficients in the $e_{G}$
basis, and quantities stored in real space use real physical values.
This means for instance that wavefunctions in the real space grid are
normalized as $\frac{|\Omega|}{N} \sum_{r} |f(r)|^{2} = 1$ where $N$
is the number of grid points.

\section{Relative and cartesian coordinates}
Let $A$ be a set of primitive lattice vectors for $\mathcal R$, and
$B$ for $\mathcal R^{*}$, with $A^{*} B = 2\pi I$. Then $x_{c} = A
x_{r}$ and $q_{c} = B q_{r}$, where the subscripts $c$ and $r$ stand
for cartesian and relative. We have
\begin{align*}
  x_{c} \cdot q_{c} = 2\pi x_{r} \cdot q_{r}
\end{align*}
\end{document}
